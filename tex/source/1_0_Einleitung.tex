\chapter{Einleitung}

\section{Motivation}

Die Motivation des Projektes liegt im Rahmen der Realisierung eines Fahrzeuges für den Hyperloop, dass mit einer Batterie und einem Motor betrieben werden soll, die Steuerung soll mittels Simulink.

\subsection{Konzept von Hyperloop}
Der Hyperloop ist ein Konzept für ein Hochgeschwindigkeitstransportsystem, das von Elon Musk populär gemacht wurde. Es besteht im Wesentlichen aus einer oder mehreren Kapseln, die sich durch fast luftleere Röhren bewegen. Die Idee ist, Reibung und Luftwiderstand, die zwei größten Hindernisse für hohe Geschwindigkeiten, zu minimieren.

Durch die Reduzierung von Luft- und Rollwiderstand können Hyperloop-Kapseln Geschwindigkeiten von über 1000 km/h erreichen. Dies ermöglicht extrem schnelle Reisen zwischen Städten, die weit voneinander entfernt sind.

Angesichts der globalen Bemühungen zur Reduzierung der CO2-Emissionen und zur Bekämpfung des Klimawandels könnte Hyperloop eine umweltfreundlichere Alternative zu Autos und Flugzeugen bieten.\newline


%Das neue Design des 48-Volt-Pods ist darauf ausgelegt, Material zu transportieren. Das Konzept sieht vor, dass eine große Lagerhalle am Stadtrand die Waren annimmt und sie dann durch das unterirdische Schienennetz zu den verschiedenen Standorten befördert. Auf diese Weise soll das Verkehrsaufkommen in den Städten durch Lastkraftwagen minimiert werden.


\section{Aufgabenstellung}

Im Rahmen dieses Projekts soll ein Pod für den Hyperloop auf 48 V dimensioniert werden.

\begin{itemize}
	\item Steuerung
	      \subitem Speedgoat
	      \subitem Matlab -> Simulink
	\item Positionsermittlung
	\item Verdrahtung
	\item Bauteile beschaffung
\end{itemize}

Die Aufgabe besteht darin, die Umsetzbarkeit einer Boardspannung von 48 V zu überprüfen. Darüber hinaus sollen die Verdrahtung und die Sensorik realisiert werden. Die Logik und die Signalverarbeitung sollen über den Speedgoat mittels Simulink durchgeführt werden.

\section{Aufbau der Arbeit}

\section{Verwandte Arbeiten}

\pagebreak

