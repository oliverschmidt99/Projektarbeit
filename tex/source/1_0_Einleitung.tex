\chapter{Einleitung}
\label{chapter:Einleitung}

\section{Motivation}
\label{section:Motivation}

Der Hyperloop ist ein innovatives Transportkonzept, das eine ökonomische, klimafreundliche und schnellere Alternative zu herkömmlichen Verkehrsmitteln wie Lastkraftwagen, Zügen und Flugzeugen bietet.
Derzeit stehen herkömmlichen Transportmitteln zwei wesentliche Hindernisse im Weg, um Personen und Güter schnell und emissionsarm zu befördern: Zum einen der hohe Luftwiderstand, der bei hohen Geschwindigkeiten den Energieverbrauch stark erhöht, und zum anderen der Rollwiderstand der Räder, der ebenfalls zu einem höheren Energiebedarf führt.
\begin{figure}[!ht]
	\begin{center}
		\includegraphics[width=1\textwidth]{img/1_strecke/strecke_1.png}
		\caption{Hyperloop der Hochschule Emden-Leer}
		\label{img_1_1:strecke}
	\end{center}
\end{figure}
\pagebreak[4]


Der Hyperloop löst diese Probleme, indem er Güter und Personen in einem Fahrzeug, das sich in einer Vakuumröhre bewegt, wie in Abbildung \ref{img_1_1:strecke} dargestellt ist, zudem wird das Fahrzeug, wie bei Magnetschwebebahntechnik angehoben, somit lassen sich Roll- und Luftwiderstand fast vollkommen aufheben.

Angesichts der globalen Bemühungen zur Reduzierung der CO2-Emissionen und zur Bekämpfung des Klimawandels könnte Hyperloop eine umweltfreundlichere Alternative zu Autos und Flugzeugen bieten.





\subsection{Institute of Hyperloop Technology}
\label{section:IHT}
Die Hochschule Emden/Leer hat im Jahr 2021 das Institut für Hyperloop-Technologie (IHT) gegründet, um aktiv an der Forschung zu dieser zukunftsweisenden Technologie teilzunehmen.

Im Rahmen dieser Forschung wurde an der Hochschule Emden eine Teststrecke mit einer Länge von 26 Metern errichtet (siehe Abbildung \ref{img_1_1:strecke}). Auf dieser Strecke soll das Fahrzeug (POD) unter realistischen Bedingungen getestet und weiterentwickelt werden.
Die Teststrecke besteht aus einem Schinensystem und einem Linarmotor. Der Linarmotor wird für die Magnetschwebebahntechnik verwendet.


Darüber hinaus engagiert sich das IHT in verschiedenen Projekten, darunter das \frqq European Hyperloop Technology Center – EuHyTeC\flqq, das europäische Hyperloop-Initiativen vernetzt und gemeinsam die nächste Generation des Transports entwickelt.
\newpage

%Das neue Design des 48-Volt-Pods ist darauf ausgelegt, Material zu transportieren. Das Konzept sieht vor, dass eine große Lagerhalle am Stadtrand die Waren annimmt und sie dann durch das unterirdische Schienennetz zu den verschiedenen Standorten befördert. Auf diese Weise soll das Verkehrsaufkommen in den Städten durch Lastkraftwagen minimiert werden.


\section{Aufgabenstellung}
\label{section:Aufgabenstellung}

\myboxy{
	\begin{itemize}
		\item Ablauforientiert erklären. Also erst die Bestellung, dann der Schaltplan und dann die Simulation mit Simulink.
		\item Aufgabenstellung in der Vergangenheit formulieren.
		\item Den Leser in der Doku struktur Einführen. Am enden in 1.x
	\end{itemize}
}{To-do}{\textwidth}


Die Motivation für dieses Projekt liegt in der Entwicklung eines Hyperloop-Fahrzeugs, das mit einer Batterie und einem Motor betrieben wird. Für die Steuerung des Fahrzeuges wurde ein echtzeitfähiges Steuerungsystem der Firma Speedgoat vorgeben, welches in Abschnitt \ref{section:speedgoat} vertieft wird.\\ \ \\

Im Rahmen des Projekts wird ein Fahrzeug (Pod) für den Hyperloop mit einer Bordspannung von 48 V konzipiert. Ziel ist es, die Machbarkeit dieser Spannung zu überprüfen und umzusetzen. Dazu gehören die Planung und Simulierung, die Integration der erforderlichen Sensorik sowie die Beschaffung der notwendigen Bauteile. Die Logik- und Signalverarbeitung wird mithilfe von Simulink auf dem echtzeitfähigen Speedgoat-System durchgeführt.
Die Steuerung erfolgt über Simulink, ein Modul von MATLAB, und umfasst die Erfassung von Position und Beschleunigung des Fahrzeuges. Der Motor wird über ein zusätzliches Steuergerät angesteuert. Die Steuerung soll als Automatensteuerung umgesetzt werden.
Die Verdrahtung des Pods wird entsprechend der Bordspannung von 48 V ausgelegt. Hierfür wird mit der Software QElectroTech ein Schaltplan erstellt.
Alle erforderlichen Bauteile für die Umsetzung der Bordspannung, die Verdrahtung und die Sensorik müssen beschafft werden.
Textergebnisse und in Betriebnahme entfallen.


\section{Aufbau der Projektdokumentation}
\label{section:Aufbau}
