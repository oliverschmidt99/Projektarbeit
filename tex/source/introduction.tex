\chapter{Einleitung}

\section{Motivation}
Im Rahmen dieses Projekts soll ein Pod für den Hyperloop auf $48$ V dimensioniert werden. Die bisherige Boardspannung von 600 V stellt sich als äußerst unvorteilhaft für Forschungszwecke heraus. Studierende sind daran gehindert, neue Experimente mit dem Pod durchzuführen, da stets eine Lebensgefahr durch die Hohespannung besteht.

Das neue Design des 48-Volt-Pods ist darauf ausgelegt, Material zu transportieren. Das Konzept sieht vor, dass eine große Lagerhalle am Stadtrand die Waren annimmt und sie dann durch das unterirdische Schienennetz zu den verschiedenen Standorten befördert. Auf diese Weise soll das Verkehrsaufkommen in den Städten durch Lastkraftwagen minimiert werden.


\section{Aufgabenstellung}

Die Aufgabe besteht darin, die Umsetzbarkeit einer Boardspannung von $48$ V zu überprüfen. Darüber hinaus sollen die Verdrahtung und die Sensorik realisiert werden. Die Logik und die Signalverarbeitung sollen über den Speedgoat mittels Simulink durchgeführt werden.

\section{Aufbau der Arbeit}

\section{Verwandte Arbeiten}

\pagebreak

